\documentclass[a4paper,10pt]{report}
\usepackage[latin1,utf8]{inputenc}
\usepackage[portuguese]{babel}
\usepackage{color}
\usepackage{graphicx}
\usepackage{ulem}
\usepackage{verbatim}
\usepackage{titlepic}
\title{A série de TV Guerra dos Tronos}
\author{Miguel Nunes}
\date{}
\titlepic {\includegraphics[scale=0.25]{GoT}}
\begin{document}
\maketitle
\chapter{Introdução}
\section{Sobre a série}
\label{sas}
A {\huge Guerra dos Tronos}\footnote{Este nome é uma marca registada} é uma super produção televisiva da HBO baseada na obra literária de Geroge R.R. Martin, é uma série {\scriptsize que redefine os parâmetros do que é possível fazer em televisão}.
\vskip 1cm
\begin{center}
\textit{Uma narrativa épica que {\Large atravessa mundos imaginários} e personagens a perder de vista.\footnote{Texto retirado de http://tv.sapo.pt/series/a-guerra-dos-tronos.}}
\end{center}
\vskip 1cm
\begin{verbatim}
UPS! Este texto não devia estar aqui / § " o % $ & ^ ~
\end{verbatim}
\section{Elenco}
\label{e}
A secção \ref{sas} falou-nos sobre a série e agora na secção \ref{e} vamos falar sobre o elenco. A tabela \ref{te} mostra-nos as personagens principais.
\begin{table}[h]
\centering
\begin{tabular}{|l|c|r|}
\hline
\textbf{Ator/Atriz} & \textbf{Personagem} & \textbf{Temporada} \\ \hline
Peter Dinklage & Tyrion Lannister & 1, 2 e 3 \\ \hline
Emilia Clarke & Daenerys Targaryen & 1, 2 e 3 \\ \hline
Richard Madden & Robb Stark & 1 e 2 \\ \hline
\end{tabular}
\caption{Principais personagens da "Guerra dos Tronos" ao longo das várias temporadas.}
\label{te}
\end{table}
\end{document}