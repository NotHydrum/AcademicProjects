\documentclass[12pt,a4paper]{article}
\usepackage[utf8]{inputenc}
\usepackage[portuguese]{babel}
\usepackage[T1]{fontenc}
\usepackage{amsmath}
\usepackage{amsfonts}
\usepackage{amssymb}
\usepackage{enumerate}
\title{Exercícios de PDT\\Folha 2 - Modo Matemático}
\date{}
\begin{document}
\maketitle
\section{Símbolos matemáticos}
\begin{enumerate}[(a)]
\item Se $L(x)$ representar "$x$ tem cabelo louro", então a frase (1) escreve-se simbolicamente $\forall x, L(x)$. A sua negação, que poderíamos coloquialmente redigir "nem todas as pessoas têm cabelo louro", escreve-se $\neg \forall x, L(x)$,e é logicamente equivalente a (2), que se escreve $\exists x, \neg L(x)$.
\item Se $A \subseteq B$ e $B \subseteq A$, então $A$ e$B$ têm exatamente os mesmos elementos e portanto, $A=B$.
\item Sejam $A = 2,3,5,7,9$, $B=1,3,5,7,9$, $C=\mathbb{N}$, $D=\{x\in \mathbb{Z}|x<5\}$
\item $\displaystyle \lim_{x \to \infty}$exp$(-x)=0$
\item Em trigonometria, a relação básica entre o seno e o coseno é conhecida como \textit{Identidade Trigonométrica Fundamental}: $cos^2 \theta +sin^2 \theta =1$
\end{enumerate}
\section{Equações}
\begin{enumerate}
\item Uma equação com duas linhas e alinhada pelo sinal de igualdade:
\begin{equation}
\begin{split}
x^2+z^3&=\sqrt{2+3y}\\
x\div 5&=z^{x+2\pi}
\end{split}
\end{equation}
\item  A equação anterior dividida em duas equações alinhadas pelo sinal de igualdade:
\begin{align}
x^2+z^3&=\sqrt{2+3y}\\
x\div 5&=z^{x+2\pi}
\end{align}
\item Três equações com três colunas (repare com atenção nos alinhamentos):
\begin{align}
X_{a}&=\sqrt{x+y} & X_{b}&=\pi +y & X_{c}&=2+x^y\\
Z_{ax}&=x^3+7 & Z_{ay}&=\sqrt{x^4+5Y}+29y & Z_{Y_{a}}=x&-2-y\\
Z_{ax}&=10 & Z_{ay}&=35y-2 & Z_{az}&=2+y
\end{align}
\item Represente o seguinte sistema:
\begin{equation}
f(x)=
\begin{cases}
1 & \text{se } x\geq 2\\
0 & \text{se } 1<x<2\\
-1 & \text{se } x\leq 1.
\end{cases}
\end{equation}
\end{enumerate}
\end{document}