\documentclass[a4paper,11pt]{beamer}
\usepackage[latin1,utf8]{inputenc}
\usepackage[portuguese]{babel}
\usepackage{color}
\usepackage{graphicx}
\usepackage{ulem}
\title{A vida do Suricata}
\author{João António-dos-Santos}
\date{17 de Novembro de 2014}
\usetheme{CambridgeUS}
\begin{document}
\begin{frame}
\maketitle
\end{frame}
\begin{frame}[allowframebreaks]{Sobre o Suricata}
%\chapter{Introdução}
\section{Sobre o Suricata}
O \textbf{suricata}, também chamado de \textbf{suricato} ou \textbf{suricate} \textit{(Suricata suricatta) é um pequeno mamífero da família Herpestidae, nativo do deserto do Kalahari. Estes animais têm cerca de meio metro de comprimento (incluindo a cauda), em média 730 gramas de peso, e pelagem acastanhada.\\
Têm garras afiadas nas patas, que lhes permitem escavar a superfície do chão e dentes afiados para penetrar nas carapaças quitinosas das suas presas. \textcolor{red}{Outra característica distinta é a sua capacidade de se elevarem nas patas traseiras, utilizando a cauda como terceiro apoio}.}
\vskip 8cm
\section{Características gerais}
\subsection{Alimentação}
\textit{Alimenta-se principalmente de insetos (cerca de 82\%), como:
\begin{itemize}
\item larvas de escaravelhos;
\item larvas de borboletas;
\item milípedes.
\end{itemize}
Alimenta-se também de:
\begin{itemize}
\item aranhas;
\item escorpiões;
\item pequenos vertebrados (répteis, anfíbios e aves);
\item ovos;
\item matéria vegetal.
\end{itemize}
\uline{São relativamente imunes ao veneno} das najas e dos escorpiões, sendo estes, inclusive, um dos alimentos que mais apreciam.}
\end{frame}
\begin{frame}{Onde avistar suricatas no habitat selvagem?}
%\chapter{Desenvolvimento}
\section{Onde avistar suricatas no habitat selvagem?}
Existem vários parques nacionais em África onde é possível avistar e até interagir com suricatas no seu habitat selvagem. No entanto, existe uma regra de ouro: os suricatas não gostam de chuva, por isso prefira dias solarengo. Em baixo apresenta-se uma lista de parques ordenada por número de suricatas por Km²: % $km^2$ não compila
\begin{description}
\item[Primeiro] Parque "Kgalagadi", África do Sul / Botswana
\item[Segundo] Parque nacional "Karoo", África do Sul
\item[Terceiro] Reserva do vale mágico do Suricata, África do Sul
\item[Quarto] Parque nacional Iona, Angola
\end{description}
\end{frame}
\begin{frame}{Subespécies}
\section{Subespécies}
Existem atualmente três subespécies de Suricata:
\begin{itemize}
\item Suricata suricatta siricata;
\item Suricata suricatta iona;
\item Suricata suricatta majoriae.
\end{itemize}
Os indivíduos de cada subespécie apresentam características distintas como se pode ver na tabela \ref{ss}.
\begin{table}
\centering
\begin{tabular}{c|c|c|c|}
\cline{2-4}
 & \textbf{Siricata} & \textbf{Iona} & \textbf{Majoriae} \\ \hline
\multicolumn{1}{|c|}{Côr do pelo} & Beje amarelo &  Castanho amarelado & Castanho escuro \\ \hline
\multicolumn{1}{|c|}{Tamanho} & 29cm & 25cm & 34cm \\ \hline
\multicolumn{1}{|c|}{Peso} & 731g & 698g & 799g \\ \hline
\end{tabular}
\caption{Características diferenciadoras entre subespécies de Suricata.}
\label{ss}
\end{table}
\end{frame}
\end{document}
